\chapter{Wnioski i dalsza praca}

Celem mojej pracy było stworzenie uniwersalnej biblioteki implementującej
podstawowe algorytmy z dziedziny wizji komputerowej. Aktualny stan pracy
pozwala na stworzenie prostych demonstracji i jest tylko zalążkiem kodu, który
mógłby być w przyszłości wykorzystywany w gotowych aplikacjach produkcyjnych.

Aby móc dokonać częściowej rekonstrukcji geometrycznej sceny z dwóch obrazów,
konieczne było znalezienie par pikseli, które (prawdopodobnie) ze sobą
korespondują. Ponieważ pojedynczy piksel jest zbyt trywialny żeby go porównywać
do czegokolwiek, musieliśmy porównywać całe regiony pikseli. W tym celu
konieczne było znajdowanie takich regionów, które są zarówno charakterystyczne
(dla danej sceny) i stabilne (niezależnie od zmiennych warunków) oraz łatwe w
znalezieniu. Rezultatem naszych obliczeń były mi. macierze przekształceń, które
pomnożone przez dowolny punkt, zwrócą korespondujący punkt na drugim obrazie.
Jest tak dlatego, ponieważ macierze te zawierają w sobie informacje o
przekształceniach, które zaszły między jednym a drugim obrazem.

Omówiona część teoretyczna jest zaledwie wstępem do szerokiego zagadnienia
wizji komputerowej.

Kolejnym elementem mojej pracy była implementacja wszystkich opisanych
algorytmów (MSER, IS-MATCH, RANSAC) w języku JavaScript. Demonstracja działania,
wraz z graficznym inferface-m wszystkich parametrów algorytmów w formie strony
HTML, jest integralną częścią tego projektu. 

Dalsze prace nad implementacją powinny być skupione przede wszystkim nad
wydajnością. Warto zauważyć, że wszystkie opisane algorytmy są trywialne w
zrównolegleniu, ale ich implementacje wciąż nie wykorzystuje potencjału wielu
rdzeni procesora.

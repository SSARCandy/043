\chapter{Maksymalnie stabilne regiony ekstremalne}

Maksymalnie Stabilne Regiony Ekstremalne (MSER) po raz pierwszy zostały opisane
w publikacji Jiri Matasa pt. "Robust Wide Baseline Stereo from Maximally Stable
Extremal Regions" i znajdują zastosowanie np. w rozpoznawaniu i śledzeniu
obiektów, składaniu fragmentów obrazu oraz trójwymiarowej rekonstrukcji
otoczenia.

\section{Charakterystyczne regiony}

Charakterystyczne regiony (ang. distinguished regions) są fragmentami obrazów,
które posiadają pewne wyróżniające właściwości pozwalające na ich powtarzalne i
trwałe wykrywanie, niezależnie od pewnych deformacji. Maksymalnie stabilny
region ekstremalny jest spójnym segmentem obrazu otrzymanym w procesie
binaryzacji przez progowanie. Właściwość "ekstremalności" odnosi się do tego,
że każdy piksel w danym rejonie jest znacznie jaśniejszy MSER+ (lub ciemniejszy
MSER-) od wszystkich pikseli otaczających region o parametr stabilności
\todo{sigma}. Ponieważ w trakcie jednego przebiegu algorytmu możemy szukać
tylko jednego typu regionów, wskazane jest powtórzenie algorytmu przetwarzacjąc 
piksele w odwrotnej kolejności. Regiony te są natomiast maksymalnie stabilne
ponieważ w procesie binaryzacji ich rozmiar, w pewnym przedziale jasności, nie
rośnie.

\begin{enumerate} MSER-y posiadają szereg pożadanych cech, w tym:
\item niewrażliwość na zmianę perspektywy, \item obojętność na zmianę
jasności (oświetlenia), \item liniowa złożoność obliczeniowa (choć w
oryginalnej publikacji złożoność prawie liniowa), \item duża dokładność
opisywanych krawędzi (w przeciwieństwie do innych algorytmów
stusujących smoothing czyli zmiękczanie krawędzi).  \end{enumerate}

Nawet po zajściu w/w transformacji, regiony, jeżeli tylko
spełnią podstawowe kryteria, będą dalej opisywać te same obszary.

\section{Algorytm}

Algorytm zaproponowany przez Davida Nistera i Henrika Steweniusa w publikacji
"Linear Time Maximally Stable Extremal Regions" jako pierwszy posiada, w
najgorszym przypadku, liniową złożoność obliczeniową. Najważniejszą zmianą w
stosunku do wcześniejszych implementacji jest kolejność w której iterujemy
piksele, dzięki czemu struktura obszarów w pamięci reprezentowana jest jako
zbiór pikseli (a nie lista połączona) W analogii, jasności pikseli tworzą pasma
górskie, a kolejność odwiedzanych pikseli determinuje woda (ze źródła
znajdującego się w dowolnym miejscu), spływająca najpierw do dolin tworząc
jeziora, później unosząc się, łącząc kolejne zbiorniki, w końcu zalewając cały
obszar (w oryginalnej implementacji poziom wody wzrasta jednomiernie na całej
powierzchni obrazu niezależnie od jego topografii).

\subparagraph{Podstawowe definicje}

Załóżmy obraz \textit{I} posiadający \textit{n} pikseli indeksowanych zmienną
\textit{x}. Każdemu pikselowi przyporządkowana jest wartość \textit{f(x)}
będąca jasnością (ang. gray-scale value) z przedziału [0..255]. Każdemu
pikselowi odpowiada także zbiór \textit{N(x)} od 2 do 4 pikseli z nim
sąsiadujących. Ścieżka z \textit{x} do \textit{y} jest ciągiem sąsiadujących ze
sobą pikseli zaczynającym się w \textit{x} i kończącym się na \textit{y}.
Spójny zbiór \textit{X} jest zbiorem pikseli posiadającymi ścieżki każdy z
każdym. Zbiór z progiem \textit{y} jest zbiorem pikseli których jasność nie
przekracza wartości y. Wraz ze zwiększaniem progu \textit{y} (zwiększając
akceptowalną jasność pikseli) powstają nowe spóje zbiory, które tworzą drzewo
komponentów (ang. component tree). Liśćmi tego drzewa są tzw. minima, czyli
komponenty o pikselach o takiej samej jasności, a węzłami komponenty powstałe
po złączeniu dwóch innych komponentów, w wyniku procesu podnoszenia progu y do
momentu w którym korzeniem drzewa stanie się komponent zawierający wszystkie
piksele obrazu.

W zależności od potrzeb, nie musimy pamiętać który piksel należy do którego
komponentu. Jeżeli w dalszym etapie analizy obrazu wystarczą nam przybliżone
położenie i kształt obszarów w postaci elips, to każdy region można opisać za
pomocą sum: \sum{1}, x, y, xx, xy, yy.

Maksymalnie stabilnymi regionami ekstremalnymi są te komponenty, któych funkcja
wzrostu osiąga lokalne minimum.

\subparagraph{Struktury danych}

Implementacja wymaga następujących struktur danych: \begin{itemize} \item maski
binarnej odwiedzonych pikseli. \item Kolejki priorytetowej granicznych pikseli
(ang. boundary pixels) nienależących do sąsiadującego komponentu, z powodu
wyższego poziomu jasności. Zostaną one zamapiętane od przetworzenia w dalszej
kolejności, poczynając od najciemniejszego. \item Stosu \textit{C} komponentów
wraz z historią jego rozmiarów (niezbędą do stwierdzenia cechy stabilności)
oraz poziomem jasności na którym aktualnie się znajduje.  \end{itemize}

\subparagraph{Przebieg algorytmu}

\begin{enumerate} \item Wyzeruj maskę binarną odwiedzonych pikseli, kolejkę
piorytetową i stos komponentów. Włóż na stos pusty komponent z poziomem
jasności większym niż jakikolwiek możliwy. \item Wybierz dowolny piksel
źródłowy (np. lewy górny róg) i zapisz go do zmiennej \textit{current\_pixel}.
Zmień jego status w masce binarnej, a jego jasność zapisz do zmiennej
\textit{current\_level}. \item Włóż na stos pusty komponent z poziomem jasności
\textit{current\_level}. \item Odwiedź pozostałe piksele sąsiadujące. Dla
każdego sąsiada sprawdź czy jest dostępny (tj. czy był odwiedzany) jeżeli nie
jest, oznacz go jako dostępny i sprawdź czy jego jasność nie jest mniejsza od
aktualnego poziomu, to dodaj go do kolejki piorytetowej. W przeciwnym razie
dodaj zawartość zmiennej \textit{current\_pixel} do kolejki, a zmiennej
przypisz sąsiada. Uaktualnij wartość zmniennej \textit{current\_level}. Przejdź
do punktu 3. \item Dodaj \textit{current\_pixel} do komponentu znajdującego się
na wierzchu stosu komponentów. \item Pobierz piksel z kolejki piorytetowej.
Jeżeli w kolejce nie ma żadnych pikseli, zakończ algorytm. Jeżeli jasność
zwróconego piksela jest równa current\_level, przejdź do punktu 4. \item
Ponieważ zrwórcony piksel znajduje się wyżej (jest jaśniejszy) od
przetwarzanego komponentu, istnieje możliwość połączenia dwóch lub więcej
komponentów w jeden. W tym celu przetwarzamy wszystkie komponenty na stosie,
których poziom jest mniejszy lub równy od obecnego. Dokonuje się tego za pomocą
podprocedury \textit{ProcessStack}. Następnie wróć do punktu 4. \end{enumerate}

Przebieg procedury \textit{ProcessStack(new\_pixel\_gray\_level)}
\begin{enumerate} \item Rozważ komponent z wierzchołka stosu. Ustaw
current\_level na mniejszą z wartości: new\_pixel\_gray\_level i poziomu
drugiego komponentu na stosie. \item Jeżeli new\_pixel\_gray\_level jest
mniejszy od poziomu drugiego komponentu na stosie, pozwól aby komponent na
wierzchu stosu miał poziom new\_pixel\_gray\_level i wyjdź z procedury. \item
Zdejmij komponent z wierzchu stosu i połącz go z drugim komponentem w kolei.
Takie połączenie traktuje się jako pojedyńczy skok rozmiaru większego z komponentów.
\item Jeżeli new\_pixel\_gray\_level jest większy od poziomu nowego komponentu
na wierzchu stosu, przejdź do punktu 1. \end{enumerate}

\todo{Czy MSER}
